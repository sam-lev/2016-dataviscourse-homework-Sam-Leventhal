\documentclass[11pt, a4paper]{article}

%MUST DOWNLOAD algorithm.sty
\usepackage{algorithm} 
%\usepackage{algorithmicx}
\usepackage{algpseudocode}
\usepackage{empheq}
\usepackage{euscript}
\usepackage{amsmath}
\usepackage{amsthm}
\usepackage{amssymb}
\usepackage{epsfig}
\usepackage{xspace}
\usepackage{color}
\usepackage{url}


\usepackage{hyperref}

%\usepackage{algpseudocode}

\usepackage{mathtools}
\DeclarePairedDelimiter\ceil{\lceil}{\rceil}
\DeclarePairedDelimiter\floor{\lfloor}{\rfloor}

%%%%%%%  For drawing trees  %%%%%%%%%
\usepackage{tikz}
\usetikzlibrary{calc, shapes, backgrounds}

%%%%%%%%%%%%%%%%%%%%%%%%%%%%%%%%%
\setlength{\textheight}{9in}
\setlength{\topmargin}{-0.600in}
\setlength{\headheight}{0.2in}
\setlength{\headsep}{0.250in}
\setlength{\footskip}{0.5in}
\flushbottom
\setlength{\textwidth}{6.5in}
\setlength{\oddsidemargin}{0in}
\setlength{\evensidemargin}{0in}
\setlength{\columnsep}{2pc}
\setlength{\parindent}{1em}
\setlength\fboxsep{1cm}
%%%%%%%%%%%%%%%%%%%%%%%%%%%%%%%%%


\newcommand{\eps}{\varepsilon}

\renewcommand{\c}[1]{\ensuremath{\EuScript{#1}}}
\renewcommand{\b}[1]{\ensuremath{\mathbb{#1}}}
\newcommand{\s}[1]{\textsf{#1}}
\newcommand*\widefbox[1]{\fbox{\hspace{2em}#1\hspace{2em}}}
\newcommand{\E}{\textbf{\textsf{E}}}
\renewcommand{\Pr}{\textbf{\textsf{Pr}}}
\renewcommand{\labelenumi}{(\alph{enumi}) }


\title{Project Proposal
}
\author{Anonymous Author(s)}
\date{}

\begin{document}

\maketitle


%%%%%%%%%%%%%%%%%%%%%%%%%%%%%%%%%%%%%%%%%%%%%%%%%%%%
%%%%%%%%%%%%%%%%%%%%%%%%%%%%%%%%%%%%%%%%%%%%%%%%%%%%
%%%%%%%%%%%%%%%%%%%%%%%%%%%%%%%%%%%%%%%%%%%%%%%%%%%%



\iffalse

%
% Image
%
\begin{figure}[H]
\centering{
\includegraphics[width=.8\linewidth]{problem1plot.png}
}
\label{fig:prob1fig}
\end{figure}

%
% Matrix
%
\[ \begin{pmatrix}
  4 & 1 \\
  0 & 4 \\
\end{pmatrix}\]

%
% Algorithm
%
  \begin{algorithm}[H]
\caption{Matrix Inversion by LU decomposition}
\begin{algorithmic}
  \For{$k = 1:n$}
      \For{$i=1:n$}
          \If{$i\neq k$}
              \State $l_{ik} \leftarrow \frac{A_{ik}}{A_{kk}}$ \Comment{(n-1)}
                    \For{$j=k+1 : 2n$}
                        \State $A_{ij} \leftarrow A_{ij} - l_{ik}A_{kj}$ \Comment{2(n-1)(2n-k)}
                    \EndFor
                    \EndIf
                    \For{$j=2n:k$}
                    \State $A_{kj} \leftarrow \frac{A_{kj}}{A_{kk}}$ \Comment{2n-k+1}
                    \EndFor
      \EndFor
  \EndFor
\end{algorithmic}
\end{algorithm}

%
% Another matrix
%
  \[
  \begin{smallmatrix}
    1 & 1 & 1 & \cdots & 1 & 2 & 3 & 4 & \cdots & n & 2 & 3 & \cdots &n\\
    1 & 2 & 3 & \cdots & n & 1 & 1 & 1 & \cdots & 1 & 2 & 3 & \cdots & n\\
    c_{1,1} & c_{1,2} & c_{1,3} & \cdots & c_{1,n} & c_{2,1} & c_{3,1} & c_{4,1} & \cdots & c_{n,1} & c_{2,2} & c_33 &\cdots &c_nn\\
  \end{smallmatrix}
\]

%
% bounding box
%
\noindent\fbox{
  \parbox{.8\textwidth}{

  }
}



\fi
%%%%%%%%%%%%%%%%%%%%%%%%%%%%%%%%%%%%%%%%%%%%%%%%%%%%%%%%%%%%%%%%%%%%%%%%%
%%%%%%%%%%%%%%%%%%%%%%%%%%%%%%%%%%%%%%%%%%%%%%%%%%%%%%%%%%%%%%%%%%%%%%%%%
%%%%%%%%%%%%%%%%%%%%%%%%%%%%%%%%%%%%%%%%%%%%%%%%%%%%%%%%%%%%%%%%%%%%%%%%%

\section{Basic Info}{. The project title, your names, e-mail addresses, UIDs, a link to the project repository.}
Project Title: An Intuitive Look on College Comparisons\\
Authors: Samuel Leventhal \& Unnikrishnan Rajagopalan\\
e-mail Addresses: Samuel ( samlev@cs.utah.edu ) Unnikrishnan (unniar@cs.utah.edu)\\
UIDs: Samuel (u0491567) Unnikrishnan (u1010114)\\
Repository: \href{https://github.com/sam-lev/2016-dataviscourse-homework-Sam-Leventhal/tree/master/project}{https://github.com/sam-lev/2016-dataviscourse-homework-Sam-Leventhal/tree/master/project}

\section{Background and Motivation}{. Discuss your motivations and reasons for choosing this project, especially any background or research interests that may have influenced your decision.}
The project proposed is the resulted from $(a)$ an interest in designing an approachable and informative means for researching potential colleges when applying and $(b)$ investigate the relation between the academic progress in a field and the funding provided to that department.

\section{Project Objectives}{. Provide the primary questions you are trying to answer with your visualization. What would you like to learn and accomplish? List the benefits.}
The backbone of the project is a visualization which allows the user to easily browse numerous colleges and compare items of interest for an applying student such as tuition cost, ranking, and so on. Of these interests we intend to obtain the number of publications produced for each school department for each year. Using rate of publications as a measure it is then our goal to rank colleges based on their level of ''academic progress''. Secondary to this, if the data is obtainable, we hope to incorporate school funding per department. We believe it will be informative to view the level of funding provided to a department and the rate of publications for that department, the hypothesis being that those with more funding contribute more to the academia. If possible it would also be interesting the source of that funding, be it private or government.
\section{Data}{. From where and how are you collecting your data? If appropriate, provide a link to your data sources.}
\begin{itemize}
  \setlength\itemsep{0.5em}
\item College names
\item College ranking
\item Cost of tuition
\item Number of Scholarships awarded with respect to number of students
\item Acceptance rate
\item Average GPA, SAT, GRE, ACT, MCAT, ect for accepted students
\item Funding per department (if possible to obtain)
\item Total publications per year for school
\item Total publications per department per year for school
\item Source of funding (if possible)
\item Quality of life for city
\item Cost of living
\item average salary
 
  \end{itemize}

\section{Data Processing}{. Do you expect to do substantial data cleanup? What quantities do you plan to derive from your data? How will data processing be implemented?}

\section{Visualization Design}{. How will you display your data? Provide some general ideas that you have for the visualization design. Develop three alternative prototype designs for your visualization. Create one final design that incorporates the best of your three designs. Describe your designs and justify your choices of visual encodings. We recommend you use the Five Design Sheet Methodology.}

\section{Must-Have Features}{. List the features without which you would consider your project to be a failure.}

\section{Optional Features}{. List the features which you consider to be nice to have, but not critical.}


\section{Project Schedule}{. Make sure that you plan your work so that you can avoid a big rush right before the final project deadline, and delegate different modules and responsibilities among your team members. Write this in terms of weekly deadlines.}

\end{document}
%%% Local Variables:
%%% mode: latex
%%% TeX-master: t
%%% End:

