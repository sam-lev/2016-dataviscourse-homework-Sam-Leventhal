\documentclass[11pt, a4paper]{article}

%MUST DOWNLOAD algorithm.sty
\usepackage{algorithm} 
%\usepackage{algorithmicx}
\usepackage{algpseudocode}
\usepackage{empheq}
\usepackage{euscript}
\usepackage{amsmath}
\usepackage{amsthm}
\usepackage{amssymb}
\usepackage{epsfig}
\usepackage{xspace}
\usepackage{color}
\usepackage{url}


\usepackage{hyperref}

%\usepackage{algpseudocode}

\usepackage{mathtools}
\DeclarePairedDelimiter\ceil{\lceil}{\rceil}
\DeclarePairedDelimiter\floor{\lfloor}{\rfloor}

%%%%%%%  For drawing trees  %%%%%%%%%
\usepackage{tikz}
\usetikzlibrary{calc, shapes, backgrounds}

%%%%%%%%%%%%%%%%%%%%%%%%%%%%%%%%%
\setlength{\textheight}{9in}
\setlength{\topmargin}{-0.600in}
\setlength{\headheight}{0.2in}
\setlength{\headsep}{0.250in}
\setlength{\footskip}{0.5in}
\flushbottom
\setlength{\textwidth}{6.5in}
\setlength{\oddsidemargin}{0in}
\setlength{\evensidemargin}{0in}
\setlength{\columnsep}{2pc}
\setlength{\parindent}{1em}
\setlength\fboxsep{1cm}
%%%%%%%%%%%%%%%%%%%%%%%%%%%%%%%%%


\newcommand{\eps}{\varepsilon}

\renewcommand{\c}[1]{\ensuremath{\EuScript{#1}}}
\renewcommand{\b}[1]{\ensuremath{\mathbb{#1}}}
\newcommand{\s}[1]{\textsf{#1}}
\newcommand*\widefbox[1]{\fbox{\hspace{2em}#1\hspace{2em}}}
\newcommand{\E}{\textbf{\textsf{E}}}
\renewcommand{\Pr}{\textbf{\textsf{Pr}}}
\renewcommand{\labelenumi}{(\alph{enumi}) }


\title{Project Proposal
}
\author{Anonymous Author(s)}
\date{}

\begin{document}

\maketitle


%%%%%%%%%%%%%%%%%%%%%%%%%%%%%%%%%%%%%%%%%%%%%%%%%%%%
%%%%%%%%%%%%%%%%%%%%%%%%%%%%%%%%%%%%%%%%%%%%%%%%%%%%
%%%%%%%%%%%%%%%%%%%%%%%%%%%%%%%%%%%%%%%%%%%%%%%%%%%%



\iffalse

%
% Image
%
\begin{figure}[H]
\centering{
\includegraphics[width=.8\linewidth]{problem1plot.png}
}
\label{fig:prob1fig}
\end{figure}

%
% Matrix
%
\[ \begin{pmatrix}
  4 & 1 \\
  0 & 4 \\
\end{pmatrix}\]

%
% Algorithm
%
  \begin{algorithm}[H]
\caption{Matrix Inversion by LU decomposition}
\begin{algorithmic}
  \For{$k = 1:n$}
      \For{$i=1:n$}
          \If{$i\neq k$}
              \State $l_{ik} \leftarrow \frac{A_{ik}}{A_{kk}}$ \Comment{(n-1)}
                    \For{$j=k+1 : 2n$}
                        \State $A_{ij} \leftarrow A_{ij} - l_{ik}A_{kj}$ \Comment{2(n-1)(2n-k)}
                    \EndFor
                    \EndIf
                    \For{$j=2n:k$}
                    \State $A_{kj} \leftarrow \frac{A_{kj}}{A_{kk}}$ \Comment{2n-k+1}
                    \EndFor
      \EndFor
  \EndFor
\end{algorithmic}
\end{algorithm}

%
% Another matrix
%
  \[
  \begin{smallmatrix}
    1 & 1 & 1 & \cdots & 1 & 2 & 3 & 4 & \cdots & n & 2 & 3 & \cdots &n\\
    1 & 2 & 3 & \cdots & n & 1 & 1 & 1 & \cdots & 1 & 2 & 3 & \cdots & n\\
    c_{1,1} & c_{1,2} & c_{1,3} & \cdots & c_{1,n} & c_{2,1} & c_{3,1} & c_{4,1} & \cdots & c_{n,1} & c_{2,2} & c_33 &\cdots &c_nn\\
  \end{smallmatrix}
\]

%
% bounding box
%
\noindent\fbox{
  \parbox{.8\textwidth}{

  }
}



\fi
%%%%%%%%%%%%%%%%%%%%%%%%%%%%%%%%%%%%%%%%%%%%%%%%%%%%%%%%%%%%%%%%%%%%%%%%%
%%%%%%%%%%%%%%%%%%%%%%%%%%%%%%%%%%%%%%%%%%%%%%%%%%%%%%%%%%%%%%%%%%%%%%%%%
%%%%%%%%%%%%%%%%%%%%%%%%%%%%%%%%%%%%%%%%%%%%%%%%%%%%%%%%%%%%%%%%%%%%%%%%%

\section{Basic Info}
Project Title: An Intuitive Look on College Comparisons\\
Authors: Samuel Leventhal \& Unnikrishnan Rajagopalan\\
e-mail Addresses: Samuel ( samlev@cs.utah.edu ) Unnikrishnan (unniar@cs.utah.edu)\\
UIDs: Samuel (u0491567) Unnikrishnan (u1010114)\\
Repository: \href{https://github.com/sam-lev/2016-dataviscourse-homework-Sam-Leventhal/tree/master/project}{https://github.com/sam-lev/2016-dataviscourse-homework-Sam-Leventhal/tree/master/project}

\section{Background and Motivation}.\textbf{ Discuss your motivations and reasons for choosing this project, especially any background or research interests that may have influenced your decision.}\\
The project proposed results from $(a)$ an interest in designing an approachable and informative means for researching potential colleges when applying for undergraduate or graduate schools and $(b)$ to develop a measure and means of comparison between schools and departments based on ``academic progress'' and other measures of achievement such as success in sports. If time permits it may also prove interesting to show the relationship between academic progress and funding. 

\section{Project Objectives}\textbf{ Provide the primary questions you are trying to answer with your visualization. What would you like to learn and accomplish? List the benefits.}
The backbone of the project is a visualization which allows the user to easily understand and compare numerous college statistics such as tuition cost, ranking, average salary after graduation and so on. Among others a college attribute of interest will be the number of publications produced for each school department within a year. Using rate of publications as a measure will allow the user to compare schools based on academic progress as well as see the change in academic progress between departments within a school over time. Secondary to this if the data is available we hope to incorporate school funding per department. We believe it will be informative to visualize the relation between funding and rate of publications. It would also be interesting the source of that funding, be it private or government.

In terms of utility the visualization will allow for an intuitive alternative to the what is now a tedious and difficult process of determining which schools are a best fit for a student moving into higher education. Much of the web crawling now required during the school selection process will be well contained in an approachable and information rich visualization in that users are able to display similar schools based on their criteria as well as compare specific schools by selection or school attributes of interest.

A second result of interest will be the ability to gauge the academic progress, tuition fees, acceptance rate, ect... of a school or schools over time. In this the development and future direction of school departments will be better estimated by the user. 

\section{Data}\textbf{ From where and how are you collecting your data? If appropriate, provide a link to your data sources.}

\begin{itemize}
  \setlength\itemsep{0.5em}
\item College Statistics (Database: \href{https://nces.ed.gov/ipeds/datacenter/}{https://nces.ed.gov/ipeds/datacenter/}
  \begin{itemize}
\item College names
\item College ranking
\item Cost of tuition
\item Number of Scholarships awarded with respect to number of students
\item Acceptance rate
\item Average GPA, SAT, GRE, ACT, MCAT, ect for accepted students
\end{itemize}
\item Funding Attributes 
\begin{itemize}
\item Funding per department.
\item Source of funding.
\end{itemize}
\item Quality Of Life Statistics
  \begin{itemize}
\item Quality of life for city
\item Cost of living
\item average salary
\end{itemize}
\item Academics Progress Statistics
  \begin{itemize}
  \item Total publications for entire school in one year
  \item Total publications for each department in one year
    \end{itemize}
\end{itemize}
  \section{Data Processing}\textbf{ Do you expect to do substantial data cleanup? What quantities do you plan to derive from your data? How will data processing be implemented?}
  There are near 1016 universities in the United States. In order to visualize these universities as circles on a map we may need to create a criteria on which schools will be included in the visualization. We may therefore need to do a basic parsing of the data to only incorporate schools satisfying these attributes. Second to this we may also need to data mine sites such as arxiv.org to organize the data by school, school department, and publication year. For funding we will also need to cluster the data in this way however also include source of funding and amount. 

  \section{Visualization Design}\textbf{ How will you display your data? Provide some general ideas that you have for the visualization design. Develop three alternative prototype designs for your visualization. Create one final design that incorporates the best of your three designs. Describe your designs and justify your choices of visual encodings. We recommend you use the Five Design Sheet Methodology.}
  The project will consist of (a) a selection bar on the left hand side where the user is able to determine which features and for what range values they would like to consider, (b) a map to it's right, (c) circled points on the map where each school is located, (d) five buttons beneath the map one of which allows the user to select specific schools to compare and the rest of which compare preset number of schools, i.e. if ten schools are selected then the ten nearest schools matching the user set criteria will be compared. Beneath will be (e) histograms demonstrating the top 10 attributes most often asked about schools such as tuition, general well being, acceptance rate, and so on. The x-axis of the histograms will be ambiguous due to the attributes not being correlated however each rectangle will be labeled as the attribute it represents. The scales of each rectangle will then be set with linear scaling whose domain is determined by the maximum values of the schools selected. And lastly, (f) beneath the histograms will be oval ranged gauges with circles representing where the selected schools are placed accordingly for all school statistics. A similar visualization can be seen on \href{https://www.oecdregionalwellbeing.org}{www.oecdregionalwellbeing.org}. 
 
  

  \section{Must-Have Features}\textbf{ List the features without which you would consider your project to be a failure.}
  Necessary visualizations and features include a map projection, hoverable and clickable points on the map for schools, legend with school statistics, click feature for each school statistic in the legend which opens a range selector in which one drags to points along a legend to represent what range to be considered, the option to select specific schools to compare, the option to compare a preset number of similar schools which fall into the user defined features of interest and their ranges, histograms for each selected school displaying it's specific attributes, a more detailed set of all features for each school consisting of oval bars representing an attribute's range and circles for the value of each selected school.

\section{Optional Features}\textbf{ List the features which you consider to be nice to have, but not critical.}


\section{Project Schedule}{. Make sure that you plan your work so that you can avoid a big rush right before the final project deadline, and delegate different modules and responsibilities among your team members. Write this in terms of weekly deadlines.}

\end{document}
%%% Local Variables:
%%% mode: latex
%%% TeX-master: t
%%% End:

